\documentclass[12pt]{article}

% set margins and spacing
\addtolength{\textwidth}{1.3in}
\addtolength{\oddsidemargin}{-.65in} %left margin
\addtolength{\evensidemargin}{-.65in}
\setlength{\textheight}{9in}
\setlength{\topmargin}{-.5in}
\setlength{\headheight}{0.0in}
\setlength{\footskip}{.375in}
\renewcommand{\baselinestretch}{1.0}
\linespread{1.0}

% load miscellaneous packages
\usepackage{csquotes}
\usepackage[american]{babel}
\usepackage[usenames,dvipsnames]{color}
\usepackage{graphicx,amsbsy,amssymb, amsmath, amsthm, MnSymbol,bbding,times, verbatim,bm,pifont,pdfsync,setspace,natbib}

% enable hyperlinks and table of contents
\usepackage[pdftex,
bookmarks=true,
bookmarksnumbered=false,
pdfview=fitH,
bookmarksopen=true,hyperfootnotes=false]{hyperref}

% define environments
\newtheorem{definition}{Definition}
\newtheorem{fact}{Fact}
\newtheorem{result}{Result}
\newtheorem{proposition}{Proposition}



\begin{document}
\title{Taxes and Tariffs}
\author{Kathryn Sarrge\thanks{Syracuse University, Economics Department. Email: kbuzard@syr.edu.} \and Dylan Thomas\thanks{abc} \and Umar Bilgrammi\thanks{abc}}
\date{\vskip-.1in \today}
\maketitle

\vskip.3in
\begin{center} {\bf Abstract} \end{center}

\begin{quote}
{\small Insert abstract text here: 75-200 words, very high-level summary of your project.}
\end{quote}

\bigskip
\section{Introduction} \label{sec:introduction}
The structure of government revenue plays a vital role in shaping a country's economic stability and growth. For developing countries, consumption taxes on goods and services (such as VAT) and tariffs on international trade are key revenue sources. Understanding the relationship between these two forms of taxation, including whether they are complementary or substitutive, can offer valuable insights for policymakers looking to optimize tax policies in search of overall national development. 

This topic is especially important because the reliance on tariffs has historically been higher in developing countries, but as these countries modernize and integrate into the global economy, we may observe a shift towards other forms of taxation, such as VAT. By analyzing this relationship, we aim to provide critical insights into how governments balance their revenue sources and the potential implications of this balance for economic development.

This study investigates the relationship between taxes on goods and services and taxes on international trade in developing countries. The primary research question is: Is there an inverse relationship between VAT revenue and tariff revenue in developing countries? Specifically, we hypothesize  if consumption tax such as VAT increases in a developed or developing country, then tariffs decrease. We will examine data from the World Integrated Trade Solution (WITS) database, which provides information on both forms of taxation as a percentage of government revenue across 194 countries from 1988 to 2022. By analyzing these two key revenue sources, we hope to better understand the fiscal strategies employed by developing nations and how these strategies have evolved over the past few decades.

In this paper, we will present a detailed analysis of the data collected from the WITS database, which includes not only taxes on goods and services and tariffs but also GDP per capita to help classify countries as developed or developing. We will start with a description of the data and its sources, followed by a theoretical discussion of the relationship between VAT and tariffs. The results section will outline the correlation between these variables and provide a series of visualizations to illustrate the data trends. Finally, the paper will conclude with a discussion of the findings, their implications for developing countries' fiscal policy, and areas for further research.

\section{Literature Review} \label{sec:literature}

This section discusses several key papers that provide insights into the relationship between tariffs, consumption taxes, and economic welfare, particularly in the context of developing countries. The papers reviewed offer different perspectives on how tax reforms like reducing tariffs while increasing consumption taxes affect government revenue and overall welfare. These studies also highlight various challenges and considerations, such as the role of informal economies and trade openness. The findings from these papers helped create the theoretical framework of this research, and help contextualize our hypothesis and results.

Michael S. Michael, Panos Hatzipanayotou, and Stephen M. Miller (1993):
This paper examines the effects of integrated reforms of tariffs and consumption taxes on welfare in less-developed countries, focusing on the conditions under which reducing tariffs and increasing consumption taxes can enhance welfare while maintaining government revenue. Their research finds that in certain contexts, such reforms can indeed improve welfare by shifting the reliance from inefficient tariffs to more efficient consumption taxes like VAT. The paper’s conclusions align with our hypothesis that tariff reduction, when combined with an increase in consumption taxes, can be beneficial for developing countries by maintaining or even improving welfare while stabilizing government revenue. This paper is useful in supporting our hypothesis when it comes to tax reform strategies in developing countries.

John Mark Hansen (1990):
Hansen's study explores the political economy of tariffs, discussing how tariffs have historically been used as a tool of advantage for governments, especially in developing countries. His analysis emphasizes that governments may rely heavily on tariffs for revenue generation, but as countries modernize, this reliance can become inefficient. The paper suggests that transitioning from tariffs to other forms of taxation, such as VAT, is necessary for long-term economic stability and growth. Hansen’s framework on the political economy of tariffs helps explain why governments in developing countries are often discouraged to reduce tariffs despite their inefficiency, which could be an obstacle in implementing the tax reforms we are studying. While his focus is on the political dynamics of tariffs, it provides a different perspective to our study by emphasizing the difficulties in shifting away from tariff-based revenue generation.

Swarnim Waglé (2011):
Waglé's research investigates the challenges and opportunities for low-income countries transitioning from trade taxes (tariffs) to domestic consumption taxes like VAT. Using data, the study models scenarios where tariffs can be reduced with minimal impact on revenue, highlighting the importance of effective domestic tax collection. Waglé’s findings suggest that the shift from trade taxes to consumption taxes is not only reasonable but also potentially advantageous if properly enforced, as it leads to a more stable and efficient revenue collection system. This is directly related to our hypothesis that increasing VAT while reducing tariffs can help maintain government revenue and improve overall economic welfare in developing countries.

Thuy Tien Ho, Xuan Hang Tran, and Quang Khai Nguyen (2023):
This paper examines the relationship between tax revenue and economic growth in developing countries, with a particular focus on the role of trade openness. The authors find that an increase in tax revenue, facilitated by greater trade openness, positively impacts economic growth. While the paper does not focus explicitly on the relationship between tariffs and VAT, it provides useful context for understanding how broader fiscal policies, including taxes, can influence the economic development of developing countries. This paper helps frame our study within the broader discussion of how tax revenue impacts growth, suggesting that our findings regarding VAT and tariffs may have further implications for economic performance. The concept of trade openness in this paper is significant as it complements our analysis of how reducing tariffs (a form of trade tax) can be beneficial when combined with an increase in consumption taxes.

M. Shahe Emran and Joseph E. Stiglitz (2005):
Emran and Stiglitz’s paper focuses on the role of indirect tax reforms in developing countries, specifically addressing the reduction of trade taxes and the increase of VAT. The authors argue that while this policy shift can theoretically enhance revenue, the presence of the informal economy complicates the effectiveness of such reforms. In practice, VAT may not capture the full tax base due to underreporting, leading to a reduction in welfare if reforms are not carefully designed. This paper offers a cautionary perspective on the benefits of reducing trade taxes and increasing VAT, suggesting that the informal economy can limit the potential gains from such reforms. Emran and Stiglitz's findings contrast with our hypothesis by emphasizing the potential negative effects of VAT reform when the informal economy is not addressed, suggesting that a more nuanced approach is necessary in our analysis of VAT and tariff reforms in developing countries.

The papers reviewed reveal a broad generalization on the potential benefits of reducing tariffs and increasing consumption taxes in developing countries, specifically when it comes to improving economic welfare and maintaining government revenue. The findings from Michael, Michael S., Panos Hatzipanayotou, and Stephen M. Miller (1993) and Waglé (2011) are largely consistent with our hypothesis that tariff reduction and VAT increases can enhance welfare while maintaining revenue. However, the political economy insights from Hansen (1990) and the concerns raised by Emran and Stiglitz (2005) add complexity to this view, suggesting that the success of such reforms depends on various factors, including political will, administrative and governmental capacity, and the structure of the informal economy, which can impact effectiveness significantly.

The findings of Ho, Thuy Tien, Xuan Hang Tran, and Quang Khai Nguyen (2023) offer a broader perspective on the relationship between trade openness and economic growth, which aligns with our study's focus on the role of trade in a country's fiscal reforms. However, our research adds specific circumstances by focusing on the direct relationship between VAT and tariffs, rather than the broader question of how trade openness influences tax revenue.

Overall, the literature provides a broad ranges of perspectives on the effectiveness and difficulties of tariff and consumption tax reforms in developing and developed countries. While most studies reviewed suggest that these reforms can greatly improve revenue and welfare of a country, some specific criteria like a country's political dynamics and informal economy mean the relationship is more complex. 

\section{Theoretical Analysis}
\label{sec:theory}
The relationship between consumption taxes like VAT and tariffs is central to understanding how developing countries structure their tax systems. Generally, VAT is considered a more stable and efficient source of revenue than tariffs, especially as countries modernize their taxation. Tariffs, which are taxes imposed on imports, are often seen as a tool for protecting domestic industries but can be disruptive to trade and economic efficiency. In contrast, consumption taxes are levied on consumption and can be easier to collect, particularly when countries have established taxation systems. As countries develop, it is expected that they may shift from relying on tariffs to more consumption-based taxes like VAT.

Our hypothesis states that there is a negative correlation between consumption taxes like VAT and tariff revenue in developing countries. Specifically, we expect that as VAT revenue increases (indicating a shift towards a modern tax system), tariff revenue will decrease. Therefore, our null hypothesis is that increases in consumption taxes like VAT have no effect on tariffs in developed or developing countries. 
The shift between consumption taxes like VAT increasing as a percent of revenue while tariffs decrease as a percent of revenue could reflect a broader trend of economic liberalization and integration into global trade, where countries increasingly adopt consumption taxes as part of their fiscal strategies and reduce their reliance on trade taxes. We also anticipate that the relationship between consumption taxes and tariffs may vary depending on a country's level of economic development, which we control for using GDP per capita.

This hypothesis is grounded in the theory of tax modernization, where governments gradually move away from potentially risky taxes like tariffs and towards more efficient and broad-based taxes consumption taxes like VAT. By examining this relationship over time, we aim to better understand the evolving fiscal landscape in developing countries.

\section{Data}
\label{sec:data}

We analyzed three time series datasets sourced from the \href{https://wits.worldbank.org/CountryProfile/en/Country/BY-COUNTRY/StartYear/1988/EndYear/2022/Indicator/GC-TAX-GSRV-VA-ZS}{World Integrated Trade Solution} website, which provides detailed information about taxes and tariffs as a percentage of revenue for developing countries, while using a third dataset including GDP per capita which helped identify what countries were considered developed or developing. These datasets span from 1988 to 2022 and cover 194 countries. The datasets we focus on represent two distinct types of revenue sources for these countries: taxes on goods and services and taxes on international trade. The third dataset is simply used for identifying countries as developed or developing based on their GDP per capita in current US dollars. 

The WITS database aggregates data from various sources. It primarily compiles national statistics on tariffs and taxes from government reports and international organizations. The data collection process involves gathering information directly from the national governments' reports, which are often based on their domestic economic activities, tax laws, and trade statistics. Additionally, the data may be supplemented by international organizations like the International Monetary Fund (IMF), World Bank, and the United Nations, who collect national statistics to ensure comparability across countries.

The collection of this data is subject to the availability and reliability of national statistical offices, and differences may arise depending on the country’s data reporting practices. While the dataset provides information for most countries, certain nations may have incomplete or missing data for specific years. In these cases, either the data is unavailable for that particular year or there are gaps due to inconsistencies in national reporting. For some countries, the data may not start until a later year, and for others, the data may be missing intermittently across the 34-year period.

In this study, we focus on developing countries, where taxes and tariffs play a crucial role in government revenue. However, there is variation in how governments rely on different revenue sources. We acknowledge that the dataset is limited to the data available from the WITS platform, and some countries may have more reliable, consistent reporting than others. For example, developed countries often have more accurate and complete records, while developing countries may face challenges in data collection or reporting.

The main variables used in our data are taxes on goods and services, taxes on international trade, and GDP per capita. Taxes on goods and services represents the revenue from taxes like VAT or sales taxes as a percentage of total revenue. It includes taxes levied on consumption, which can vary greatly across countries, especially in the form of Value Added Taxes (VAT) or other consumption-based taxes. Taxes on international trade represents the revenue from tariffs as a percentage of total revenue, reflecting the importance of international trade in a country's economic structure. Finally, GDP per capita represents a country's gross domestic product divided by its population, which can be considered a decent indicator of the standard of living in a country. In our project specifically, GDP per capita will be used to identify developing and developed countries. 

The summary statistics for the three variables- taxes on goods and services, taxes on international trade—are presented in the table below. This table provides an overview of the mean, standard deviation, minimum, and maximum values for these variables across all countries and years in the dataset.


These statistics reflect the general range of taxes and tariffs as a percentage of government revenue. It is important to note that these figures represent averages across developing countries, with significant variation between individual countries.




\subsection{Survey data}

\section{Results}
\label{sec:result}

We have conducted three separate analysis on our final data, all done in Stata. The first analysis we conducted was a correlation test to in relation to our hypothesis to determine whether our two variables had a positive or negative correlation. Our current working hypothesis is if Value added tax or sales tax revenue increases in a developing country, then tariff revenue decreases. Our null hypothesis is that in developing countries, there is no difference in revenue when value added tax or tariffs change. The specific code we ran is pwcorr international vat, which uses our two variables to determine there correlation. We got a value of -.3250 for this test, meaning these variables have a moderate strength negative correlation. This shows that there is a weak to moderate negative relationship between tariffs and value added tax in developing countries. Thus, we reject the null hypothesis. We plan on conducting further testing in order to truly determine the validity of our hypothesis. Additionally, we created a scatter plot, bar graph, and histogram to visualize our data. These figures help accurately compare our data, but we plan an altering aspects of these figures to better resemble to results of our testing. 



\section{Discussion}
\label{sec:discussion}

Optional. This is where you would discuss any of the following
\begin{itemize}
    \item caveats (are there problems with the data that there are no obvious ways to resolve? if so, how might this impact
    \item future work / next steps
    \item implications of the results: that is, how your findings -- if they were causally identified -- might inform policymaking, etc.
\end{itemize}

\section{Conclusion}
\label{sec:conclusion}

Re-state (in different words) what you did and what you learned. If your discussion (Section 6) would be short, you can just have a Conclusion section that includes your discussion (that is, leave out a separate Discussion section).

\newpage
\section*{Bibliography}
\singlespacing
\setlength\bibsep{0pt}

\begin{itemize}
\item Michael, Michael S., Panos Hatzipanayotou, and Stephen M. Miller. "Integrated reforms of tariffs and consumption taxes." Journal of Public Economics 52.3 (1993): 417-428.
\item Hansen, John Mark. “Taxation and the Political Economy of the Tariff.” International Organization 44.4 (1990): 527–551.
\item Waglé, Swarnim. "Coordinating tax reforms in the poorest countries: Can lost tariffs be recouped?." World Bank Policy Research Working Paper 5919 (2011).
\item Ho, Thuy Tien, Xuan Hang Tran, and Quang Khai Nguyen. "Tax revenue-economic growth relationship and the role of trade openness in developing countries." Cogent Business and Management 10(2) (2023)
\item M. Shahe Emran, Joseph E. Stiglitz, "On selective indirect tax reform in developing countries" Journal of Public Economics 599-623 (2005)
\end{itemize}










\newpage
\section*{Data Appendix} \label{sec:appendixa}
\addcontentsline{toc}{section}{Appendix A}

You should at least direct your reader to your replication package. You might put key elements of your replication package in this section as well.

\end{document}
